\documentclass[review]{elsarticle}

\usepackage{amsmath}
\usepackage{lineno,hyperref}
\modulolinenumbers[5]
\usepackage{subcaption}
\usepackage{nicefrac}

\journal{}

%%%%%%%%%%%%%%%%%%%%%%%
%% Elsevier bibliography styles
%%%%%%%%%%%%%%%%%%%%%%%
%% To change the style, put a % in front of the second line of the current style and
%% remove the % from the second line of the style you would like to use.
%%%%%%%%%%%%%%%%%%%%%%%

%% Numbered
%\bibliographystyle{model1-num-names}

%% Numbered without titles
%\bibliographystyle{model1a-num-names}

%% Harvard
%\bibliographystyle{model2-names.bst}\biboptions{authoryear}

%% Vancouver numbered
%\usepackage{numcompress}\bibliographystyle{model3-num-names}

%% Vancouver name/year
%\usepackage{numcompress}\bibliographystyle{model4-names}\biboptions{authoryear}

%% APA style
%\bibliographystyle{model5-names}\biboptions{authoryear}

%% AMA style
%\usepackage{numcompress}\bibliographystyle{model6-num-names}

%% `Elsevier LaTeX' style
%\bibliographystyle{elsarticle-num}
\bibliographystyle{elsarticle-num-names}
\makeatletter
\providecommand{\doi}[1]{%
  \begingroup
  \let\bibinfo\@secondoftwo
  \urlstyle{rm}%
  \href{http://dx.doi.org/#1}{%
    doi:\discretionary{}{}{}%
    \nolinkurl{#1}%
  }%
  \endgroup
}
\makeatother
%%%%%%%%%%%%%%%%%%%%%%%

\begin{document}

\begin{frontmatter}

\title{Performance implications of modified wavenumber analysis for the piecewise parabolic method}
%\tnotetext[mytitlenote]{Fully documented templates are available in the elsarticle package on \href{http://www.ctan.org/tex-archive/macros/latex/contrib/elsarticle}{CTAN}.}

%% Group authors per affiliation:
% \author{Elsevier\fnref{myfootnote}}
% \address{Radarweg 29, Amsterdam}
% \fntext[myfootnote]{Since 1880.}

% %% or include affiliations in footnotes:
% \author[mymainaddress]{Marc T. Henry de Frahan}
% \cortext[mycorrespondingauthor]{Corresponding author}
% \ead{marc.henrydefrahan@nrel.gov}

% \author[mymainaddress]{Ray W. Grout}

% \address[mymainaddress]{High Performance Algorithms and Complex Fluids, Computational Science, National Renewable Energy Laboratory, 15013 Denver W Pkwy, ESIF301, Golden, CO 80401}

\begin{abstract}
TBD
\end{abstract}

\begin{keyword}
TBD \sep TBD
\end{keyword}

\end{frontmatter}

\linenumbers

\section{Introduction}

% In this work, we\dots

% This paper is organized as follows\dots

Findings:
\begin{itemize}
\item PPM spatial discretization has a dependence on the CFL number, $C = \nicefrac{a\Delta t}{h}$ (this is alluded to in the original PPM paper)
\item The dispersion and diffusion errors from modified wavenumber analysis are functions of $C$ and $hk$
\item This implies that one can tune $C$ and $hk$ to achieve a target error while minimizing simulation time
\end{itemize}

Question: the theory (as detailled below) shows this but does it hold up in practice? Is there a mistake?

\section{Modified wavenumber analysis}\label{sec:}

In this work, we perform modified wavenumber analysis of the piecewise
parabolic method (PPM)~\cite{Colella1984}, a second order finite
volume method with parabolic solution profile reconstruction. The
modified wavenumber analysis is performed for the linear advection
equation,
\begin{align}
  \label{eq:linear_advection}
  \frac{\partial u}{\partial t} + a \frac{\partial u}{\partial x} = 0,
\end{align}
subject to the initial condition $u(x,0) = u_0(x)$, and the advection
velocity, $a$ is positive, without loss of generality. The solution is
discretized according to finite volume methodologies and the cell
average value of the solution at time $t^n$ in cell $j$ is
\begin{align}
  \label{eq:fv}
  u^n_j = \frac{1}{\Delta x_j} \int_{x_{j-\nicefrac{1}{2}}}^{x_{j+\nicefrac{1}{2}}} u(x,t) \mathrm{d}x
\end{align}
where $\Delta x_j = x_{j+\nicefrac{1}{2}} - x_{j-\nicefrac{1}{2}}$ is
the cell length, and $x_{j-\nicefrac{1}{2}}$ and
$x_{j+\nicefrac{1}{2}}$ are the left and right cell edges. We will
assume a uniform grid in this work, $\Delta x_j = h~\forall~j$. The
semi-discrete form of the linear advection equation becomes
\begin{align}
  \label{eq:discrete_advection}
  \frac{\mathrm{d} u_j}{\mathrm{d} t} =  \frac{a}{h} (f_{j-\nicefrac{1}{2}} - f_{j+\nicefrac{1}{2}}),
\end{align}
where $f_{j-\nicefrac{1}{2}}$ and $f_{j+\nicefrac{1}{2}}$ are the cell
interface fluxes. These are defined by the PPM as,
\begin{align}
  \label{eq:fluxes}
  f_{j+\nicefrac{1}{2}} &= u_{R,j} - \frac{1}{2} \frac{a \Delta t}{h} \left( \Delta u_j - \left( 1-\frac{2}{3} \frac{a \Delta t}{h} \right) u_{6,j} \right),\\
  f_{j-\nicefrac{1}{2}} &= u_{R,j-1} - \frac{1}{2} \frac{a \Delta t}{h} \left( \Delta u_{j-1} - \left( 1-\frac{2}{3} \frac{a \Delta t}{h} \right) u_{6,j-1} \right),
\end{align}
where, assuming $a>0$,
\begin{align*}
  u_{R,j} &= u_{j+\nicefrac{1}{2}} = \frac{7}{12} (u_j^n + u_{j+1}^n) - \frac{1}{12} (u_{j+2}^n + u_{j-1}^n),\\
  u_{R,j-1} &= u_{j-\nicefrac{1}{2}} = \frac{7}{12} (u_{j-1}^n + u_{j}^n) - \frac{1}{12} (u_{j+1}^n + u_{j-2}^n),\\
  \Delta u_j &= u_{R,j} - u_{L,j} = u_{R,j} - u_{R,j-1}\\
  \Delta u_{j-1} &= u_{R,j-1} - u_{R,j-2}\\
  u_{6,j} &= 6 \left( u_j^n - \frac{1}{2} (u_{R,j-1} + u_{R,j}) \right)\\
  u_{6,j-1} &= 6 \left( u_{j-1}^n - \frac{1}{2} (u_{R,j-2} + u_{R,j-1}) \right).
\end{align*}
One of the particularities of this method is the appearance of the
Courant number, $C = \nicefrac{a\Delta t}{h}$, in the spatial
discretization formulation. This arises from integrating the parabolic
solution profile across the cell interfaces over the time step
$\Delta t$. As noted in the original PPM paper, this discretization
results in a second order accurate method which becomes fourth order
accurate in the limit of $\Delta t \to 0$~\cite{Colella1984}. We
explore this explicit interaction between the time step and spatial
discretizations on the solution accuracy using a modified wavenumber
analysis.

Let $u(x,t) = \exp{\left( i(kx-\omega t) \right)}$, where $k$ is the
wavenumber, $\omega$ is the angular frequency, and
$k = \nicefrac{\omega}{a} = \overline{\omega}$ satisfies
Equation\,(\ref{eq:linear_advection}). Using the discrete version of
this solution and substituting into the discrete partial differential
equation, Equation\,(\ref{eq:discrete_advection}), we obtain an
expression for $\overline{\omega}^*(kh)$, the discrete approximation
to $\overline{\omega}$ that provides estimates of the dispersion,
$\Re (h\overline{\omega}^*)$, and diffusion,
$\Im (h\overline{\omega}^*)$, errors due to the numerical method:
\begin{align*}
  & \frac{h}{a} \frac{\mathrm{d} u_j}{\mathrm{d} t} = f_{j-\nicefrac{1}{2}} - f_{j+\nicefrac{1}{2}},\\
  \Leftrightarrow & -i h \overline{\omega}^* u_j = \Delta u_j,\\
  \Leftrightarrow & -i h \overline{\omega}^* = \Delta.
\end{align*}

For PPM, the spatial discretization operator is
\begin{align*}
  -ih\overline{\omega}^* = & \left(- \frac{C^{2}}{12} + \frac{C}{12}\right) e^{- 3 i h k} + \left(\frac{7 C^{2}}{12} - \frac{C}{2}- \frac{1}{12}\right) e^{- 2 i h k} + \left(- \frac{4 C^{2}}{3} - \frac{5 C}{3} + \frac{2}{3}\right) e^{- i h k}\\
  & + \frac{4 C^{2}}{3} - \frac{7 C}{3} + \left(- \frac{7 C^{2}}{12} + \frac{5 C}{4} - \frac{2}{3}\right) e^{i h k} + \left(\frac{C^{2}}{12} - \frac{C}{6} + \frac{1}{12}\right) e^{2 i h k}.
\end{align*}
The Taylor expansion of the real and imaginary parts of $h\overline{\omega}^*$ are
\begin{align*}
  \Re (h\overline{\omega}^*) &= hk - \frac{C^2}{6} (hk)^3 + \left( \frac{C^2}{24} + \frac{C}{12} - \frac{1}{30} \right) (hk)^5 + \mathcal{O}\left((hk)^6\right),\\
  \Im (h\overline{\omega}^*) &= -\frac{C}{2} (hk)^2 + \left( \frac{C^2}{12} - \frac{C}{24} \right) (hk)^4 + \mathcal{O}\left((hk)^6\right).
\end{align*}
As $\Delta t \to 0$ ($C \to 0$), the method becomes fourth order
accurate. The normalized dispersion error, $\epsilon$, is
\begin{align*}
  \epsilon =\frac{hk - \Re (h\overline{\omega}^*)}{\pi} = \frac{1}{\pi} &\left( hk + \left( \frac{3C^2}{4} - \frac{5C}{12} - \frac{4}{3} \right) \sin(hk) \right.\\
                                                                        &\left. + \left( -\frac{C}{2} + \frac{C}{3} + \frac{1}{6} \right) \sin(2hk)\right.\\
                                                                        &\left. + \left( \frac{C^2}{12} - \frac{C}{12} \right) \sin(3hk)\right)
\end{align*}
and normalized diffusion error, $\gamma$, is
\begin{align*}
  \gamma = \frac{\Im (h\overline{\omega}^*)}{2} = \frac{1}{2}  &\left( \frac{4C^2}{3} - \frac{7C}{3} + \left( -\frac{23C^2}{12} + \frac{35C}{12} \right) \cos(hk) \right.\\
                                                               &\left. + \left( \frac{2C^2}{3} - \frac{2C}{3} \right) \cos(2hk)\right.\\
                                                                        &\left. + \left( -\frac{C^2}{12} + \frac{C}{12} \right) \cos(3hk)\right)
\end{align*}

Figure\,\ref{fig:temporal_analysis} illustrates the dispersion and
diffusion errors dependence on $C$. For $C=1$, the PPM method reduces
to a standard second-order finite volume method. As $C$ decreases, the
diffusion and dispersion errors decrease.

\begin{figure}[!tbp]%
  \centering%
  \begin{subfigure}[t]{0.32\textwidth}%
    \includegraphics[width=\textwidth]{./figs/temporal_wavenumber.pdf}%
    \caption{Modified wavenumber.}\label{fig:dispersion}%
  \end{subfigure}%
  \hfill%
  \begin{subfigure}[t]{0.32\textwidth}%
    \includegraphics[width=\textwidth]{./figs/temporal_dispersion.pdf}%
    \caption{Dispersion.}\label{fig:dispersion}%
  \end{subfigure}%
  \hfill%
  \begin{subfigure}[t]{0.32\textwidth}%
    \includegraphics[width=\textwidth]{./figs/temporal_diffusion.pdf}%
    \caption{Diffusion.}\label{fig:diffusion}%
  \end{subfigure}%
  \caption{Modified wavenumber analysis.}\label{fig:temporal_analysis}%
\end{figure}%

For three-dimensional simulations, the simulation time, $w$ scales as
$\Delta t$ and $h^3$,
\begin{align*}
  w \sim \frac{1}{\Delta t~h^3} = \frac{k^3}{\Delta t (hk)^3} = \frac{u k^4}{C (hk)^4}.
\end{align*}
Let 
\begin{align*}
 \tau = \frac{w}{\overline{w}} = \frac{\pi^4}{C(hk)^4},
\end{align*}
be the normalized simulation time, where
$\overline{w}=\nicefrac{u k^4}{\pi^4}$. We can formulate a constrained
optimization problem that seeks to minimize the simulation time given
a target error
\begin{align*}
  &\min_{C,h} \tau\\
  \text{ s.t. }& C \in [0,1]\\
  & kh \in [0,\pi]\\
  & e = \alpha
\end{align*}
where $e = \pi\epsilon + 2 \gamma$ is the total error and $\alpha$ is
the target error ($\alpha \in [0, \pi+2]$). The solution to the
optimization problem is presented in
Figure\,\ref{fig:total_error}. The optimal path given a target error
implies that the same error can be achieved faster when using $C=0.07$
and $hk=\nicefrac{\pi}{4}$ than when using $C=0.4$ and
$hk=\nicefrac{\pi}{8}$.

\begin{figure}[!tbp]%
  \centering%
  \includegraphics[width=0.8\textwidth]{./figs/total_error.pdf}%
  \caption{Total error, $e$, as function of $C$ and $hk$. Solid black: path for minimum simulation time given a target error; dashed grey: isocontours of $e$.}\label{fig:total_error}%
\end{figure}%


\section{Conclusion}\label{sec:ccl}
This is a very odd finding. What's missing or wrong? Can this be found in the wild (i.e. actual simulation)?

\section*{Acknowledgments}
This research was supported by the Exascale Computing Project (ECP), Project Number: 17-SC-20-SC, a collaborative effort of two DOE organizations -- the Office of Science and the National Nuclear Security Administration -- responsible for the planning and preparation of a capable exascale ecosystem -- including software, applications, hardware, advanced system engineering, and early testbed platforms -- to support the nation's exascale computing imperative.

\section*{References}

\bibliography{library}

\end{document}