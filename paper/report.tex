\documentclass[review]{elsarticle}

\usepackage{amsmath}
\usepackage{lineno,hyperref}
\modulolinenumbers[5]
\usepackage{subcaption}
\usepackage{nicefrac}

\journal{}

%%%%%%%%%%%%%%%%%%%%%%%
%% Elsevier bibliography styles
%%%%%%%%%%%%%%%%%%%%%%%
%% To change the style, put a % in front of the second line of the current style and
%% remove the % from the second line of the style you would like to use.
%%%%%%%%%%%%%%%%%%%%%%%

%% Numbered
%\bibliographystyle{model1-num-names}

%% Numbered without titles
%\bibliographystyle{model1a-num-names}

%% Harvard
%\bibliographystyle{model2-names.bst}\biboptions{authoryear}

%% Vancouver numbered
%\usepackage{numcompress}\bibliographystyle{model3-num-names}

%% Vancouver name/year
%\usepackage{numcompress}\bibliographystyle{model4-names}\biboptions{authoryear}

%% APA style
%\bibliographystyle{model5-names}\biboptions{authoryear}

%% AMA style
%\usepackage{numcompress}\bibliographystyle{model6-num-names}

%% `Elsevier LaTeX' style
%\bibliographystyle{elsarticle-num}
\bibliographystyle{elsarticle-num-names}
\makeatletter
\providecommand{\doi}[1]{%
  \begingroup
  \let\bibinfo\@secondoftwo
  \urlstyle{rm}%
  \href{http://dx.doi.org/#1}{%
    doi:\discretionary{}{}{}%
    \nolinkurl{#1}%
  }%
  \endgroup
}
\makeatother
%%%%%%%%%%%%%%%%%%%%%%%

\begin{document}

\begin{frontmatter}

\title{Performance implications of modified wavenumber analysis for the piecewise parabolic method}
%\tnotetext[mytitlenote]{Fully documented templates are available in the elsarticle package on \href{http://www.ctan.org/tex-archive/macros/latex/contrib/elsarticle}{CTAN}.}

%% Group authors per affiliation:
% \author{Elsevier\fnref{myfootnote}}
% \address{Radarweg 29, Amsterdam}
% \fntext[myfootnote]{Since 1880.}

%% or include affiliations in footnotes:
\author[mymainaddress]{Marc T. Henry de Frahan}
\cortext[mycorrespondingauthor]{Corresponding author}
\ead{marc.henrydefrahan@nrel.gov}

\author[mymainaddress]{Ray W. Grout}

\address[mymainaddress]{High Performance Algorithms and Complex Fluids, Computational Science, National Renewable Energy Laboratory, 15013 Denver W Pkwy, ESIF301, Golden, CO 80401}

\begin{abstract}
TODO
\end{abstract}

\begin{keyword}
TBD \sep TBD
\end{keyword}

\end{frontmatter}

\linenumbers

\section{Introduction}

In this work, we\dots

This paper is organized as follows\dots

\section{Modified wavenumber analysis}\label{sec:}

In this work, we perform modified wavenumber analysis of the piecewise
parabolic method (PPM)~\cite{Colella1984}, a second order finite
volume method with parabolic solution profile reconstruction. The
modified wavenumber analysis is performed for the linear advection
equation,
\begin{align}
  \label{eq:linear_advection}
  \frac{\partial u}{\partial t} + a \frac{\partial u}{\partial x} = 0,
\end{align}
subject to the initial condition $u(x,0) = u_0(x)$, and the advection
velocity, $a$ is positive, without loss of generality. The solution is
discretized according to finite volume methodologies and the cell
average value of the solution at time $t^n$ in cell $j$ is
\begin{align}
  \label{eq:fv}
  u^n_j = \frac{1}{\Delta x_j} \int_{x_{j-\nicefrac{1}{2}}}^{x_{j+\nicefrac{1}{2}}} u(x,t) \mathrm{d}x
\end{align}
where $\Delta x_j = x_{j+\nicefrac{1}{2}} - x_{j-\nicefrac{1}{2}}$ is
the cell length, and $x_{j-\nicefrac{1}{2}}$ and
$x_{j+\nicefrac{1}{2}}$ are the left and right cell edges. We will
assume a uniform grid in this work, $\Delta x_j = h~\forall~j$. The
discretized linear advection becomes
\begin{align}
  \label{eq:discrete_advection}
  u^{n+1}_j = u^n_j + a \frac{\Delta t}{h} (f_{j-\nicefrac{1}{2}} - f_{j+\nicefrac{1}{2}}),
\end{align}
where $t^{n+1} = t^n + \Delta t$ and $f_{j-\nicefrac{1}{2}}$ and
$f_{j+\nicefrac{1}{2}}$ are the cell interface fluxes. These are
defined by the PPM as,
\begin{align}
  \label{eq:fluxes}
  f_{j+\nicefrac{1}{2}} &= u_{R,j} - \frac{1}{2} \frac{a \Delta t}{h} \left( \Delta u_j - \left( 1-\frac{2}{3} \frac{a \Delta t}{h} \right) u_{6,j} \right),\\
  f_{j-\nicefrac{1}{2}} &= u_{R,j-1} - \frac{1}{2} \frac{a \Delta t}{h} \left( \Delta u_{j-1} - \left( 1-\frac{2}{3} \frac{a \Delta t}{h} \right) u_{6,j-1} \right),
\end{align}
where, assuming $a>0$,
\begin{align*}
  u_{R,j} &= u_{j+\nicefrac{1}{2}} = \frac{7}{12} (u_j^n + u_{j+1}^n) - \frac{1}{12} (u_{j+2}^n + u_{j-1}^n),\\
  u_{R,j-1} &= u_{j-\nicefrac{1}{2}} = \frac{7}{12} (u_{j-1}^n + u_{j}^n) - \frac{1}{12} (u_{j+1}^n + u_{j-2}^n),\\
  \Delta u_j &= u_{R,j} - u_{L,j} = u_{R,j} - u_{R,j-1}\\
  \Delta u_{j-1} &= u_{R,j-1} - u_{R,j-2}\\
  u_{6,j} &= 6 \left( u_j^n - \frac{1}{2} (u_{R,j-1} + u_{R,j}) \right)\\
  u_{6,j-1} &= 6 \left( u_{j-1}^n - \frac{1}{2} (u_{R,j-2} + u_{R,j-1}) \right).
\end{align*}
One of the particularities of this method is the appearance of the
Courant number, $C = \nicefrac{a\Delta t}{h}$, in the spatial
discretization formulation. This arises from integrating the parabolic
solution profile across the cell interfaces over the time step
$\Delta t$. As noted in the original PPM paper, this discretization
results in a second order accurate method which becomes fourth order
accurate in the limit of $\Delta t \to 0$~\cite{Colella1984}. We
explore this explicit interaction between the time step and spatial
discretizations on the solution accuracy using a modified wavenumber
analysis.

Let $u(x,t) = \exp{\left( i(kx-\omega t) \right)}$, where $k$ is the
wavenumber, $\omega$ is the angular frequency, and
$k = \nicefrac{\omega}{a} = \overline{\omega}$ satisfies
Equation\,(\ref{eq:linear_advection}). Using the discrete version of
this solution and substituting into the discrete partial differential
equation, Equation\,(\ref{eq:discrete_advection}), we obtain an
expression for $\overline{\omega}^*(kh)$, the discrete approximation
to $\overline{\omega}$ that provides estimates of the dispersion,
$\Re (\overline{\omega}^* h)$, and diffusion,
$\Im (\overline{\omega}^* h)$, errors due to the numerical method.

% compare the numerical performance of two codes in
% simulating the decay of homogeneous isotropic turbulence.

% A spectral code was used to perform a direct numerical simulation of
% forced homogeneous isotropic turbulence at a Taylor microscale
% Reynolds number of $Re_\lambda= \nicefrac{u_0' \lambda_0}{\nu}=133$,
% where $u_0' = \sqrt{\nicefrac{\overline{u_i u_i}}{3}}$ is the initial
% mean fluctuating velocity,
% $\lambda_0 = \sqrt{\nicefrac{15 \nu}{\epsilon_0}}~u_0'$ is the initial
% Taylor microscale, $\epsilon_0$ is the initial dissipation rate, and
% $\nu$ is the kinematic viscosity. For the spectral code simulations,
% $u_0'=\sqrt{2}$, $\epsilon_0=1.2$, and $\nu=0.0028$. The spectral code
% solves the non-dimensional incompressible Navier-Stokes equations in
% the vorticity formulation~\cite{Kim1987} using Fourier-Galerkin
% spatial discretization. Time integration is performed with low-storage
% explicit third order Runge-Kutta~\cite{Spalart1991}. The forcing used
% to reach the steady state solution applies to large scales and
% produces a specified energy injection rate~\cite{Mohan2017}. The
% wavenumber resolution for the spectral code was $256^3$, at which
% point the solution is converged. The steady state solution is used as
% an initial condition for the decay of homogeneous isotropic turbulence
% in both the spectral code and a compressible finite volume code,
% PeleC.

% PeleC is an explicit compressible Navier-Stokes flow solver based on
% the AMReX library CITE. The spatial discretization is based on the
% piecewise parabolic method, a second-order finite volume
% method~\cite{Colella1984}. Time integration is performed using second
% order Runge-Kutta. PeleC solves the compressible Navier-Stokes
% equations in conservative form,
% \begin{align*}
%   &\frac{\partial\rho}{\partial t} + \frac{\partial }{\partial x_j} \left( \rho u_j \right) = 0,\\
%   &\frac{\partial}{\partial t} \left( \rho u_i \right) + \frac{\partial}{\partial x_j} \left(\rho u_i u_j + p \delta_{ij} -\sigma_{ij} \right) = 0,\\
%   &\frac{\partial}{\partial t} \left( \rho E \right) + \frac{\partial}{\partial x_j} \left( \left( \rho E+p \right) u_j + q_j - \sigma_{ij} u_i\right) = 0
% \end{align*}
% where $\rho$ is the density, $u_j$ is the velocity for the $x_j$
% direction, $p$ is the pressure, $E = e + \frac{u_i u_i}{2}$ is the
% total energy, $e = c_v T$ is the internal energy, $T$ is the
% temperature, and $c_v$ is the heat capacity at constant volume. The ideal
% gas equation of state with a ratio of specific heats, $\gamma$, set to
% 1.4, is used to relate the thermodynamic variables. Additionally, the diffusive fluxes are
% \begin{align*}
%   &\sigma_{ij} = 2\mu S_{ij} - \frac{2}{3}\mu \delta_{ij} S_{kk}\\
%   &q_j = -k \frac{\partial T}{\partial x_j}  
% \end{align*}
% where
% $S_{ij} = \frac{1}{2}\left(\frac{\partial u_i}{\partial x_j} +
%   \frac{\partial u_j}{\partial x_i} \right)$ is the strain-rate
% tensor, $\mu$ is the dynamic viscosity, and $k$ is the thermal
% conductivity. The turbulent Mach number, $M_t = \nicefrac{u'}{c_s}$,
% where $u'$ is the mean fluctuating velocity field and $c_s$ is the
% speed of sound, was initially set to $0.1$ to ensure a weakly
% compressible flow and used to define the initial $u_0'$. The reference
% temperature, $T_0$, and pressure, $p_0$, are 300K and 1
% atmosphere. The Prandtl number, $Pr = \nicefrac{\mu c_p}{k}$, where
% $c_p$ is the heat capacity at constant pressure, is set to $0.71$. The
% Taylor microscale Reynolds number was set to match the spectral code
% and used to define the dynamic viscosity. The three-dimensional domain
% ranges from $[0,L]$, where $L=2\pi$, with periodic boundary
% conditions. In both codes, the homogeneous isotropic turbulence is
% decayed for approximately $20\tau$, where
% $\tau=\nicefrac{\lambda_0}{u_0'}$.

% To compute the initial condition for PeleC, the velocity fields from
% the spectral code are integrated to compute a cell average in each
% finite volume cell:
% \begin{align*}
%   \overline{u}_j^k = \frac{1}{V_k} \int_{V_k} u_j \mathrm{d}V_k
% \end{align*}
% where $V_k$ denotes the volume of cell $k$, and $\overline{u}_j^h$ is
% the cell averaged solution in that cell. This integral is computed in
% each cell using a three-dimensional second order gaussian quadrature
% rule. To ensure an accurate representation of the flow field in the
% finite volume discretization, we use an inverse non-uniform fast
% Fourier transform to interpolate the spectral solution at each
% quadrature node. For this work, a Python wrapper of the NUFFT library,
% an $\mathcal{O}(N\log(N))$ implementation of the non-uniform
% FFT~\cite{Greengard2004}, was extended to three-dimensional functions
% to perform the gaussian quadrature
% interpolation.\footnote{\url{https://github.com/dfm/python-nufft}}
% This projection procedure was performed for all the uniform finite
% volume grids used in this study: $32^3$, $64^3$, $128^3$, $256^3$,
% $512^3$, $768^3$, and $1024^3$ cells. While the spectral flow
% field is divergence-free, its projection on the finite volume grid is
% not strictly divergence-free, though the divergence converges to zero
% as the mesh is refined. This projection procedure ensures that an
% accurate representation of the spectral solution is recovered in the
% finite volume discretization space.

% \section{Decay of homogeneous isotropic turbulence}\label{sec:}
% In this section, we compare the temporal evolution of various metrics
% from both the spectral code and PeleC. We denote
% $\langle \cdot \rangle = \frac{1}{L^3} \int \cdot~\mathrm{d}\Omega$
% the metrics that are volume averaged in the domain.

% The decay of the mean kinetic energy,
% $E_k = \nicefrac{\langle u_i u_i \rangle}{2}$ normalized by
% $E_{k,0} = \nicefrac{3 u'^2_0}{2}$, is shown in
% Figure\,\ref{fig:KE}. The kinetic energy decays to approximately 20\%
% the initial value. For this quantity, simulations are converged at
% $512^3$ and match the spectral simulation. Low resolution simulations
% exhibit a sharp initial decrease in kinetic energy when the initial
% small scales dissipate due to numerical diffusion.

% The dilatation indicates that the initial velocity fields are not
% divergence-free as noted previously, though it decreases to zero as
% the resolution increases, Figure\,\ref{fig:dilatation}. For the
% converged simulations, the small sharp initial increase in dilatation
% indicates energy entering the acoustic modes. However, the dilatation
% decreases thereafter and remains small throughout the simulations,
% indicating that the simulations are essentially incompressible.

% Temperature fluctuations, $T'=T-T_0$, illustrate that the simulations
% remained approximately isothermal in time,
% Figure\,\ref{fig:var_temp}. Temperature fluctuations are converged at
% $512^3$ and remain very small at all times.

% For lower resolution cases, the dissipation rate, calculated in
% spectral space as $\epsilon = 2 \nu \sum k^2 E(k)$, exhibits a sharp
% drop initially due to missing high wavenumber scales. As this is a
% metric based on solution derivatives, convergence is only obtained at
% $768^3$, instead of $512^3$ as reported for other metrics such as
% kinetic energy, and matches the spectral solution at that
% resolution. For converged solutions, the dissipation rate decreases
% smoothly in time as smaller scales are dissipated and the inertial
% range decreases. Due to the relationship between enstrophy and
% dissipation rate for homogeneous flows, the temporal evolution of the
% enstrophy exhibits the same conclusions.

% Normalized velocity derivative moments,
% $\nicefrac{\left\langle \left( \frac{\partial u_i}{\partial
%         x_i}\right)^n \right\rangle}{\left\langle \left(
%       \frac{\partial u_i}{\partial x_i}\right)^2
%   \right\rangle^{\frac{n}{2}}}$, are shown in
% Figure\,\ref{fig:ho_stats}, for the skewness ($n=3$) and the kurtosis
% ($n=4$). Skewness is not converged at $768^3$ though it exhibits a
% similar behavior as $1024^3$ until $t=15\tau$. The value of the
% skewness for the higher resolution simulations is around $-0.5$, as
% observed in other studies of turbulent
% flows~\cite{Sreenivasan1997}. Kurtosis exhibits similar trends and
% increasingly deviates from the Gaussian value of 3 as the resolution
% increases, Figure\,\ref{fig:kurtosis}.

% \begin{figure}[!tbp]%
%   \centering%
%   \begin{subfigure}[t]{0.48\textwidth}%
%     \includegraphics[width=\textwidth]{./figs/KE.pdf}%
%     \caption{Kinetic energy.}\label{fig:KE}%
%   \end{subfigure}%
%   \hfill%
%   \begin{subfigure}[t]{0.48\textwidth}%
%     \includegraphics[width=\textwidth]{./figs/dilatation.pdf}%
%     \caption{Dilatation, $\theta = \nicefrac{\partial u_i}{\partial x_i}$.}\label{fig:dilatation}%
%   \end{subfigure}\\%
%   \begin{subfigure}[t]{0.48\textwidth}%
%     \includegraphics[width=\textwidth]{./figs/var_temp.pdf}%
%     \caption{Temperature fluctuations.}\label{fig:var_temp}%
%   \end{subfigure}%
%   \hfill%
%   \begin{subfigure}[t]{0.48\textwidth}%
%     \includegraphics[width=\textwidth]{./figs/dissipation.pdf}%
%     \caption{Dissipation rate.}\label{fig:dissipation}%
%   \end{subfigure}%
%   \caption{Temporal evolution of metrics. Solid red: $32^3$; dashed
%     green: $64^3$; dot-dashed blue: $128^3$; dotted orange: $256^3$;
%     dot-dot-dashed purple: $512^3$; dotted burgundy: $768^3$; dotted
%     magenta: $1024^3$; dashed black: spectral code.}\label{fig:metrics}%
% \end{figure}%

% \begin{figure}[!tbp]%
%   \centering%
%   \begin{subfigure}[t]{0.48\textwidth}%
%     \includegraphics[width=\textwidth]{./figs/skewness.pdf}%
%     \caption{Skewness.}\label{fig:skewness}%
%   \end{subfigure}%
%   \hfill%
%   \begin{subfigure}[t]{0.48\textwidth}%
%     \includegraphics[width=\textwidth]{./figs/kurtosis.pdf}%
%     \caption{Kurtosis.}\label{fig:kurtosis}%
%   \end{subfigure}%
%   \caption{Temporal evolution of higher order statistics. Solid red: $32^3$; dashed
%     green: $64^3$; dot-dashed blue: $128^3$; dotted orange: $256^3$;
%     dot-dot-dashed purple: $512^3$; dotted burgundy: $768^3$; dotted
%     magenta: $1024^3$.}\label{fig:ho_stats}%
% \end{figure}%

% % The mean fluctuating velocity decreases steadily over time as the
% % eddies are dissipated. For this quantity, the simulations are
% % converged at $512^3$. Low resolution simulations exhibit a sharp
% % initial decrease in $u'$ when the initial small scales dissipate due
% % to numerical diffusion.
% % \begin{figure}[!tbp]%
% %   \centering%
% %   \includegraphics[width=0.45\textwidth]{./figs/urms.pdf}%
% %   \caption{Velocity fluctuations as function of time. Solid red:
% %     $32^3$; dashed green: $64^3$; dot-dashed blue: $128^3$; dotted
% %     orange: $256^3$; dot-dot-dashed purple: $512^3$; dotted burgundy:
% %     $768^3$; dotted magenta: $1024^3$; dashed black: spectral code.}\label{fig:urms}%
% % \end{figure}%

% \subsection{Energy spectrum}

% The three-dimensional energy spectrum is defined as
% \begin{align*}
%   E(k) = \frac{1}{2} \int_S \phi_{ii}(k_0,k_1,k_2) \mathrm{d}S(k),
% \end{align*}
% where $k=\sqrt{k_0^2 + k_1^2 + k_2^2}$,
% $\phi_{ii}(k_0,k_1,k_2) = \hat{u}_i(k_0,k_1,k_2)
% \hat{u}_i(k_0,k_1,k_2)$ and is filtered so that only valid wavenumber
% combinations are counted, and $\hat{\cdot}$ denotes the spectral
% space. The integral is approximated by averaging
% $\phi_{ii}(k_0,k_1,k_2)$ over a binned $k$ and multiplying by the
% surface area of the sphere at $k$. The bins are defined by rounding
% the wavenumber $k$ to the closest integer. An average of every $k$ in
% each bin is then used as the reported value for that bin. The spectra
% are calculated for the spectral code and PeleC simulations at
% different resolutions.

% Initial spectra and those at $t=5\tau$ and $10\tau$ are shown in
% Figure\,\ref{fig:E3D}. The $768^3$ PeleC simulation presents a
% converged energy spectrum and matches the spectral data. Energy at
% wavenumbers larger than $k=128$ stays small and no energy pile-up is
% observed. Lower resolution simulations exhibit significant
% underpredictions of the energy at higher wavenumbers, due to numerical
% diffusion acting at those scales. In order to present clearly the
% inertial and dissipation range, the spectra are normalized according
% to~\citet{Kerr1990}, Figure\,\ref{fig:enorm}. For the higher
% resolution simulations, the initial inertial range is about one decade
% and the normalized spectra is approximately 2, as noted
% by~\citet{Kerr1990}, Figure\,\ref{fig:enorm0}. The dissipation range
% starts at approximately $\eta k = 0.4$, where $\eta$ is the Kolmogorov
% length scale, and is constant through the simulation time. The
% inertial range itself decreases as a function of time as evidenced by
% the decrease in the normalized spectra at low $\eta k$,
% Figure\,\ref{fig:enorm10}.

% \begin{figure}[!tbp]%
%   \centering%
%   \begin{subfigure}[t]{0.32\textwidth}%
%     \includegraphics[width=\textwidth]{./figs/E3D0.pdf}%
%     \caption{$t = 0\tau$.}\label{fig:E3D1}%
%   \end{subfigure}%
%   \hfill%
%   \begin{subfigure}[t]{0.32\textwidth}%
%     \includegraphics[width=\textwidth]{./figs/E3D5.pdf}%
%     \caption{$t = 5\tau$.}\label{fig:E3D5}%
%   \end{subfigure}%
%   \hfill%
%   \begin{subfigure}[t]{0.32\textwidth}%
%     \includegraphics[width=\textwidth]{./figs/E3D10.pdf}%
%     \caption{$t = 10\tau$.}\label{fig:E3D10}%
%   \end{subfigure}%
%   \caption{Three-dimensional energy spectra. Solid red: $32^3$; dashed
%     green: $64^3$; dot-dashed blue: $128^3$; dotted orange: $256^3$;
%     dot-dot-dashed purple: $512^3$; dotted burgundy: $768^3$; dotted
%     magenta: $1024^3$; dashed black: spectral code.}\label{fig:E3D}%
% \end{figure}%

% \begin{figure}[!tbp]%
%   \centering%
%   \begin{subfigure}[t]{0.32\textwidth}%
%     \includegraphics[width=\textwidth]{./figs/enorm0.pdf}%
%     \caption{$t = 0\tau$.}\label{fig:enorm0}%
%   \end{subfigure}%
%   \hfill%
%   \begin{subfigure}[t]{0.32\textwidth}%
%     \includegraphics[width=\textwidth]{./figs/enorm5.pdf}%
%     \caption{$t = 5\tau$.}\label{fig:enorm5}%
%   \end{subfigure}%
%   \hfill%
%   \begin{subfigure}[t]{0.32\textwidth}%
%     \includegraphics[width=\textwidth]{./figs/enorm10.pdf}%
%     \caption{$t = 10\tau$.}\label{fig:enorm10}%
%   \end{subfigure}%
%   \caption{Normalized three-dimensional energy spectra according
%   to~\cite{Kerr1990}. Solid red: $32^3$; dashed green: $64^3$;
%   dot-dashed blue: $128^3$; dotted orange: $256^3$; dot-dot-dashed
%   purple: $512^3$; dotted burgundy: $768^3$; dotted magenta:
%   $1024^3$; dashed black: spectral code.}\label{fig:enorm}%
% \end{figure}%

% \section{Conclusion}\label{sec:ccl}
% In this work, we performed numerical comparisons of a spectral code
% with PeleC, a second order finite volume code for the compressible
% Navier-Stokes equations. For integrated metrics such as kinetic
% energy, PeleC simulations are converged at 4 points per wavelength or
% twice the spectral code resolution. For higher order statistics such
% as dissipation rate and spectra, simulations are converged at 6 points
% per wavelength or three times the spectral code resolution.

% This work, including data sets, demonstration notebooks, analysis
% scripts, and figures, can be publicly accessed at CITE URL.

\section*{Acknowledgments}
This research was supported by the Exascale Computing Project (ECP), Project Number: 17-SC-20-SC, a collaborative effort of two DOE organizations -- the Office of Science and the National Nuclear Security Administration -- responsible for the planning and preparation of a capable exascale ecosystem -- including software, applications, hardware, advanced system engineering, and early testbed platforms -- to support the nation's exascale computing imperative.

\section*{References}

\bibliography{library}

\end{document}